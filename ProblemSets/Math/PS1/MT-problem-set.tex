\documentclass[letterpaper,12pt]{article}
\usepackage{array}
\usepackage{threeparttable}
\usepackage{geometry}
\geometry{letterpaper,tmargin=1in,bmargin=1in,lmargin=1.25in,rmargin=1.25in}
\usepackage{fancyhdr,lastpage}
\pagestyle{fancy}
\lhead{}
\chead{}
\rhead{}
\lfoot{}
\cfoot{}
\rfoot{\footnotesize\textsl{Page \thepage\ of \pageref{LastPage}}}
\renewcommand\headrulewidth{0pt}
\renewcommand\footrulewidth{0pt}
\usepackage[format=hang,font=normalsize,labelfont=bf]{caption}
\usepackage{listings}
\lstset{frame=single,
  language=Python,
  showstringspaces=false,
  columns=flexible,
  basicstyle={\small\ttfamily},
  numbers=none,
  breaklines=true,
  breakatwhitespace=true
  tabsize=3
}
\usepackage{amsmath}
\usepackage{amssymb}
\usepackage{amsthm}
\usepackage{harvard}
\usepackage{setspace}
\usepackage{float,color}
\usepackage[pdftex]{graphicx}
\usepackage{hyperref}
\hypersetup{colorlinks,linkcolor=black,urlcolor=blue}
\theoremstyle{definition}
\newtheorem{theorem}{Theorem}
\newtheorem{acknowledgement}[theorem]{Acknowledgement}
\newtheorem{algorithm}[theorem]{Algorithm}
\newtheorem{axiom}[theorem]{Axiom}
\newtheorem{case}[theorem]{Case}
\newtheorem{claim}[theorem]{Claim}
\newtheorem{conclusion}[theorem]{Conclusion}
\newtheorem{condition}[theorem]{Condition}
\newtheorem{conjecture}[theorem]{Conjecture}
\newtheorem{corollary}[theorem]{Corollary}
\newtheorem{criterion}[theorem]{Criterion}
\newtheorem{definition}[theorem]{Definition}
\newtheorem{derivation}{Derivation} % Number derivations on their own
\newtheorem{example}[theorem]{Example}
\newtheorem{exercise}[theorem]{Exercise}
\newtheorem{lemma}[theorem]{Lemma}
\newtheorem{notation}[theorem]{Notation}
\newtheorem{problem}[theorem]{Problem}
\newtheorem{proposition}{Proposition} % Number propositions on their own
\newtheorem{remark}[theorem]{Remark}
\newtheorem{solution}[theorem]{Solution}
\newtheorem{summary}[theorem]{Summary}
%\numberwithin{equation}{section}
\bibliographystyle{aer}
\newcommand\ve{\varepsilon}
\newcommand\boldline{\arrayrulewidth{1pt}\hline}


\begin{document}

\begin{flushleft}
  \textbf{\large{Problem Set Measure Theory}} \\
  OSELab2019\\
  Luca Maria Schüpbach
\end{flushleft}
\\I'm extremely sorry for not being able to finish more of this problem set. I tried, got help, then failed, (in loop) and decided at some point to focus on the other topics of this week.

\vspace{5mm}

\noindent\textbf{Problem 1.3}

Following the definitions we can see that: $\mathcal{G}_{1}$ is not a ($\sigma$-)algebra as  $\mathcal{G}_{1}$ is not closed under complements. $\mathcal{G}_{2}$ is an algebra but not a $\sigma$-algebra. $\mathcal{G}_{3}$ is an algebra and a $\sigma$-algebra.

\vspace{5mm}
\noindent\textbf{Problem 1.7}

The power set is the `largest' possible $\sigma$-algebra because it includes all possible subsets of X.
\\ As $\mathcal{A}$ is an algebra, we know that $\varnothing$ $\in \mathcal{A}$. For the second requirement to hold it must be true that:$\varnothing^c= X-\varnothing^c = X \in\mathcal{A}$. Thus, by fullfilling the definition of an algebra, we know that $\{\varnothing, X\} \subset \mathcal{A}$, which is the `smallest' possible $\sigma$-algebra.

\vspace{5mm}
\noindent\textbf{Problem 1.10}

To proof that the intersection is also a $\sigma$-algebra we have to proof the three conditions that need to be met. \\
\\ 1) $\emptyset \in \cap_{\alpha} S_{\alpha}:$\\ As $\left\{\mathcal{S}_{\alpha}\right\}$ is a family of $\sigma$-algebras, by definition, $\varnothing\in \mathcal{S}_{\alpha}$ holds for each $\sigma$-algebra in the family. Thus, also for the intersection. \\
\\ 2) Let $A \in \cap_{\alpha} S_{\alpha}\;\forall\;\alpha$. We know by definition: $A \in S_{\alpha}\;\forall\;\alpha$. As $S_{\alpha}$ is a $\sigma$-algebra, it must hold that  $A^c\in S_{\alpha}\;\forall\;\alpha$. Hence, this also holds for the family of $\sigma$-algebras: $A^c \in \cap_{\alpha}$ $S_{\alpha}\;\forall\;\alpha\implies\cap_{\alpha} S_{\alpha}$
is closed under complements and finite unions.\\
\\ 3) ...

\vspace{5mm}
\noindent\textbf{Problem 1.22}
\\ Monotone:
\\ As A is a subset of B and by the definition of a measure:
$\mu({B})=\mu({B \cap A^c})+\mu({A})$. As the measure is set to be non-negative, this implies that $\mu({A})\leq\mu({B}) = \lambda({A})\leq\mu({B})$ \\
\\Countably subadditive:
\\
...


\vspace{5mm}
\noindent\textbf{Problem 1.23} \\i) $\lambda$ is defined such that $\lambda(A)=\mu(A \cap B)$. If $A=0 \implies \lambda(\emptyset)=\mu(\emptyset \cap B)=\mu(\emptyset)$
\\ ii) By the definition of a measure assuming that: $\left\{A_{i}\right\}_{i=1}^{\infty} \subset \mathcal{S} \text { s.t. } A_{i} \cap A_{j}=\emptyset \forall i \neq j$ holds. Then $\mu(\cup_{i=1}^{\infty} A_i) = \mu((\cup_{i=1}^{\infty} A_i) \cap B)$
Since the $A_i$'s are disjoint, we can re-write: $\mu(\cup_{i=1}^{\infty}(A_i \cap B))=\sum_{i=1}^{\infty} \mu(A_{i} \cap B)$

\vspace{5mm}
\noindent\textbf{Problem 1.26}
Let $A=\cap_{n=1}^{\infty} A_{n}$ and $B=\cup_{n=1}^{\infty} B_{n}$. This is an increasing sequence $\implies \left\{B_{n}\right\}_{n=1}^{\infty}$

Hence:
\begin{equation*}
  \mu\left(A_{1}-A\right) = \mu\left(B)\right = \mu(\cup_{n=1}^{\infty} B_{n})
  =\lim _{n \rightarrow \infty} \mu\left(B_{n}\right) =\mu\left(A_{1}\right)-
  \lim _{n\rightarrow \infty} \mu\left(A_{n}\right))
\end{equation*}

\vspace{5mm}
\noindent\textbf{Problem 2.10}

B can be written as: $B=(B \cap E) \cup\left(B \cap E^{c}\right)$.
By countable subadditivity $\mu^{*}(B) \leq \mu^{*}(B \cap E)+\mu^{*}\left(B \cap E^{c}\right)$ and $\mu^{*}(B) \geq \mu^{*}(B \cap E)+\mu^{*}\left(B \cap E^{c}\right)$. Hence, $\mu^{*}(B) = \mu^{*}(B \cap E)+\mu^{*}\left(B \cap E^{c}\right)$ holds.

\vspace{5mm}
\noindent\textbf{Problem 2.14}


\vspace{5mm}
\noindent\textbf{Problem 3.1}

Let $A=\left(a_{1}, a_{2}, \ldots a_{n}\right)$ be a countable set. Let $a \in R$, then $a \subset[a-\epsilon, a+\epsilon]$, then $\lambda(a) \leq \lambda(a-\epsilon, a+\epsilon)=2 \epsilon$ for all $\epsilon > 0$. Hence, $\lambda(a)=0$ and as $A=\left(a_{1}, a_{2}, \ldots a_{n}\right)$ is a countable set, we know that: $\lambda(A)=0$

\vspace{5mm}
\noindent\textbf{Problem 3.7}

To show:

$$\{x \in X : f(x)<a\}$$

can be replaced with any of the following:

$$ \begin{array}{l}{\{x \in X : f(x) \leq a\}} \\ {\{x \in X : f(x)>a\}} \\ {\{x \in X : f(x) \geq a\}}\end{array}$$

We know that: $\{x \in X : f(x)<a\} \in \mathcal{M}$ and closed under complements. $\implies f^{-1}([a, \infty))=\left(f^{-1}(-\infty, a)\right)^{c}$ hence, $f^{-1}([a, \infty)) \in \mathcal{M}$.
Therefore, we know that: $\{x \in X : f(x)<a\} = \{x \in X : f(x) \leq a\}$.
\\

Next, we use that $f^{-1}((a, \infty))=\cap_{n=1}^{\infty} f^{-1}\left(\left(a-\frac{1}{n}, \infty\right)\right)$. As $\mathcal{M}$ is closed under countable intersection, we have $f^{-1}(a, \infty) \in \mathcal{M}$.  Hence, we know that: $\{x \in X : f(x) \leq a\}= \{x \in X : f(x)>a\}$

Analogue procedure for the last one.


\end{document}
